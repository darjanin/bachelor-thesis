% Generated by Sphinx.
\def\sphinxdocclass{report}
\documentclass[a4paper,12pt,oneside]{sphinxmanual}
\usepackage[utf8]{inputenc}
% \usepackage{graphicx}
\usepackage{pdfpages}
\DeclareUnicodeCharacter{00A0}{\nobreakspace}
\usepackage[T1]{fontenc}
\usepackage[english,slovak]{babel}
\usepackage{times}

\usepackage{longtable}
\usepackage{sphinx}
\usepackage{multirow}
\setcounter{tocdepth}{3} \linespread{1.5} \usepackage[top=2.5cm,bottom=2.5cm,right=2cm,left=3.5cm]{geometry}

\title{Tutorial to Compass and SASS Documentation}
\date{May 15, 2013}
\release{0.3}
\author{Milan Darjanin}
\newcommand{\sphinxlogo}{}
\renewcommand{\releasename}{Release}
\makeindex

\makeatletter
\def\PYG@reset{\let\PYG@it=\relax \let\PYG@bf=\relax%
    \let\PYG@ul=\relax \let\PYG@tc=\relax%
    \let\PYG@bc=\relax \let\PYG@ff=\relax}
\def\PYG@tok#1{\csname PYG@tok@#1\endcsname}
\def\PYG@toks#1+{\ifx\relax#1\empty\else%
    \PYG@tok{#1}\expandafter\PYG@toks\fi}
\def\PYG@do#1{\PYG@bc{\PYG@tc{\PYG@ul{%
    \PYG@it{\PYG@bf{\PYG@ff{#1}}}}}}}
\def\PYG#1#2{\PYG@reset\PYG@toks#1+\relax+\PYG@do{#2}}

\expandafter\def\csname PYG@tok@\endcsname{\def\PYG@tc##1{\textcolor[rgb]{0.97,0.97,0.95}{##1}}}
\expandafter\def\csname PYG@tok@vc\endcsname{\def\PYG@tc##1{\textcolor[rgb]{0.97,0.97,0.95}{##1}}}
\expandafter\def\csname PYG@tok@gs\endcsname{\let\PYG@bf=\textbf}
\expandafter\def\csname PYG@tok@cm\endcsname{\def\PYG@tc##1{\textcolor[rgb]{0.46,0.44,0.37}{##1}}}
\expandafter\def\csname PYG@tok@vg\endcsname{\def\PYG@tc##1{\textcolor[rgb]{0.97,0.97,0.95}{##1}}}
\expandafter\def\csname PYG@tok@m\endcsname{\def\PYG@tc##1{\textcolor[rgb]{0.68,0.51,1.00}{##1}}}
\expandafter\def\csname PYG@tok@l\endcsname{\def\PYG@tc##1{\textcolor[rgb]{0.68,0.51,1.00}{##1}}}
\expandafter\def\csname PYG@tok@ge\endcsname{\let\PYG@it=\textit}
\expandafter\def\csname PYG@tok@il\endcsname{\def\PYG@tc##1{\textcolor[rgb]{0.68,0.51,1.00}{##1}}}
\expandafter\def\csname PYG@tok@cs\endcsname{\def\PYG@tc##1{\textcolor[rgb]{0.46,0.44,0.37}{##1}}}
\expandafter\def\csname PYG@tok@cp\endcsname{\def\PYG@tc##1{\textcolor[rgb]{0.46,0.44,0.37}{##1}}}
\expandafter\def\csname PYG@tok@ni\endcsname{\def\PYG@tc##1{\textcolor[rgb]{0.97,0.97,0.95}{##1}}}
\expandafter\def\csname PYG@tok@ld\endcsname{\def\PYG@tc##1{\textcolor[rgb]{0.90,0.86,0.45}{##1}}}
\expandafter\def\csname PYG@tok@nl\endcsname{\def\PYG@tc##1{\textcolor[rgb]{0.97,0.97,0.95}{##1}}}
\expandafter\def\csname PYG@tok@nn\endcsname{\def\PYG@tc##1{\textcolor[rgb]{0.97,0.97,0.95}{##1}}}
\expandafter\def\csname PYG@tok@no\endcsname{\def\PYG@tc##1{\textcolor[rgb]{0.40,0.85,0.94}{##1}}}
\expandafter\def\csname PYG@tok@na\endcsname{\def\PYG@tc##1{\textcolor[rgb]{0.65,0.89,0.18}{##1}}}
\expandafter\def\csname PYG@tok@nb\endcsname{\def\PYG@tc##1{\textcolor[rgb]{0.97,0.97,0.95}{##1}}}
\expandafter\def\csname PYG@tok@nc\endcsname{\def\PYG@tc##1{\textcolor[rgb]{0.65,0.89,0.18}{##1}}}
\expandafter\def\csname PYG@tok@nd\endcsname{\def\PYG@tc##1{\textcolor[rgb]{0.65,0.89,0.18}{##1}}}
\expandafter\def\csname PYG@tok@ne\endcsname{\def\PYG@tc##1{\textcolor[rgb]{0.65,0.89,0.18}{##1}}}
\expandafter\def\csname PYG@tok@nf\endcsname{\def\PYG@tc##1{\textcolor[rgb]{0.65,0.89,0.18}{##1}}}
\expandafter\def\csname PYG@tok@nx\endcsname{\def\PYG@tc##1{\textcolor[rgb]{0.65,0.89,0.18}{##1}}}
\expandafter\def\csname PYG@tok@si\endcsname{\def\PYG@tc##1{\textcolor[rgb]{0.90,0.86,0.45}{##1}}}
\expandafter\def\csname PYG@tok@s2\endcsname{\def\PYG@tc##1{\textcolor[rgb]{0.90,0.86,0.45}{##1}}}
\expandafter\def\csname PYG@tok@vi\endcsname{\def\PYG@tc##1{\textcolor[rgb]{0.97,0.97,0.95}{##1}}}
\expandafter\def\csname PYG@tok@py\endcsname{\def\PYG@tc##1{\textcolor[rgb]{0.97,0.97,0.95}{##1}}}
\expandafter\def\csname PYG@tok@nt\endcsname{\def\PYG@tc##1{\textcolor[rgb]{0.98,0.15,0.45}{##1}}}
\expandafter\def\csname PYG@tok@nv\endcsname{\def\PYG@tc##1{\textcolor[rgb]{0.97,0.97,0.95}{##1}}}
\expandafter\def\csname PYG@tok@s1\endcsname{\def\PYG@tc##1{\textcolor[rgb]{0.90,0.86,0.45}{##1}}}
\expandafter\def\csname PYG@tok@sh\endcsname{\def\PYG@tc##1{\textcolor[rgb]{0.90,0.86,0.45}{##1}}}
\expandafter\def\csname PYG@tok@ow\endcsname{\def\PYG@tc##1{\textcolor[rgb]{0.98,0.15,0.45}{##1}}}
\expandafter\def\csname PYG@tok@mf\endcsname{\def\PYG@tc##1{\textcolor[rgb]{0.68,0.51,1.00}{##1}}}
\expandafter\def\csname PYG@tok@bp\endcsname{\def\PYG@tc##1{\textcolor[rgb]{0.97,0.97,0.95}{##1}}}
\expandafter\def\csname PYG@tok@mh\endcsname{\def\PYG@tc##1{\textcolor[rgb]{0.68,0.51,1.00}{##1}}}
\expandafter\def\csname PYG@tok@c1\endcsname{\def\PYG@tc##1{\textcolor[rgb]{0.46,0.44,0.37}{##1}}}
\expandafter\def\csname PYG@tok@o\endcsname{\def\PYG@tc##1{\textcolor[rgb]{0.98,0.15,0.45}{##1}}}
\expandafter\def\csname PYG@tok@kc\endcsname{\def\PYG@tc##1{\textcolor[rgb]{0.40,0.85,0.94}{##1}}}
\expandafter\def\csname PYG@tok@c\endcsname{\def\PYG@tc##1{\textcolor[rgb]{0.46,0.44,0.37}{##1}}}
\expandafter\def\csname PYG@tok@kr\endcsname{\def\PYG@tc##1{\textcolor[rgb]{0.40,0.85,0.94}{##1}}}
\expandafter\def\csname PYG@tok@err\endcsname{\def\PYG@tc##1{\textcolor[rgb]{0.59,0.00,0.31}{##1}}\def\PYG@bc##1{\setlength{\fboxsep}{0pt}\colorbox[rgb]{0.12,0.00,0.06}{\strut ##1}}}
\expandafter\def\csname PYG@tok@kd\endcsname{\def\PYG@tc##1{\textcolor[rgb]{0.40,0.85,0.94}{##1}}}
\expandafter\def\csname PYG@tok@ss\endcsname{\def\PYG@tc##1{\textcolor[rgb]{0.90,0.86,0.45}{##1}}}
\expandafter\def\csname PYG@tok@sr\endcsname{\def\PYG@tc##1{\textcolor[rgb]{0.90,0.86,0.45}{##1}}}
\expandafter\def\csname PYG@tok@mo\endcsname{\def\PYG@tc##1{\textcolor[rgb]{0.68,0.51,1.00}{##1}}}
\expandafter\def\csname PYG@tok@mi\endcsname{\def\PYG@tc##1{\textcolor[rgb]{0.68,0.51,1.00}{##1}}}
\expandafter\def\csname PYG@tok@kn\endcsname{\def\PYG@tc##1{\textcolor[rgb]{0.98,0.15,0.45}{##1}}}
\expandafter\def\csname PYG@tok@sx\endcsname{\def\PYG@tc##1{\textcolor[rgb]{0.90,0.86,0.45}{##1}}}
\expandafter\def\csname PYG@tok@n\endcsname{\def\PYG@tc##1{\textcolor[rgb]{0.97,0.97,0.95}{##1}}}
\expandafter\def\csname PYG@tok@p\endcsname{\def\PYG@tc##1{\textcolor[rgb]{0.97,0.97,0.95}{##1}}}
\expandafter\def\csname PYG@tok@s\endcsname{\def\PYG@tc##1{\textcolor[rgb]{0.90,0.86,0.45}{##1}}}
\expandafter\def\csname PYG@tok@kp\endcsname{\def\PYG@tc##1{\textcolor[rgb]{0.40,0.85,0.94}{##1}}}
\expandafter\def\csname PYG@tok@w\endcsname{\def\PYG@tc##1{\textcolor[rgb]{0.97,0.97,0.95}{##1}}}
\expandafter\def\csname PYG@tok@kt\endcsname{\def\PYG@tc##1{\textcolor[rgb]{0.40,0.85,0.94}{##1}}}
\expandafter\def\csname PYG@tok@sc\endcsname{\def\PYG@tc##1{\textcolor[rgb]{0.90,0.86,0.45}{##1}}}
\expandafter\def\csname PYG@tok@sb\endcsname{\def\PYG@tc##1{\textcolor[rgb]{0.90,0.86,0.45}{##1}}}
\expandafter\def\csname PYG@tok@k\endcsname{\def\PYG@tc##1{\textcolor[rgb]{0.40,0.85,0.94}{##1}}}
\expandafter\def\csname PYG@tok@se\endcsname{\def\PYG@tc##1{\textcolor[rgb]{0.68,0.51,1.00}{##1}}}
\expandafter\def\csname PYG@tok@sd\endcsname{\def\PYG@tc##1{\textcolor[rgb]{0.90,0.86,0.45}{##1}}}

\def\PYGZbs{\char`\\}
\def\PYGZus{\char`\_}
\def\PYGZob{\char`\{}
\def\PYGZcb{\char`\}}
\def\PYGZca{\char`\^}
\def\PYGZam{\char`\&}
\def\PYGZlt{\char`\<}
\def\PYGZgt{\char`\>}
\def\PYGZsh{\char`\#}
\def\PYGZpc{\char`\%}
\def\PYGZdl{\char`\$}
\def\PYGZhy{\char`\-}
\def\PYGZsq{\char`\'}
\def\PYGZdq{\char`\"}
\def\PYGZti{\char`\~}
% for compatibility with earlier versions
\def\PYGZat{@}
\def\PYGZlb{[}
\def\PYGZrb{]}
\makeatother

\begin{document}

\def\school{Comenius University, Bratislava}
\def\faculty{Faculty of Mathematics, Physics and Informatics}
\def\title{Tutorial to SASS and Compass}
\def\thesis{Bachelor Thesis}
\def\author{Milan Darjanin}
\def\year{2013}
\def\placeandyear{Bratislava, \year}
\def\supervisor{RNDr. Tomáš Kulich, PhD.\ }
\def\studyprogramme{Computer Science}
\def\studyfield{2508 Computer Science, Informatics}
\def\department{Department of Computer Science}

\pdfinfo{/Author (\author) /Title (\title)}

\selectlanguage{english}
\frontmatter
%OBAL
\include{frontmatter/cover}

%TITULNY LIST
\include{frontmatter/title}

\shorthandoff{-} %docasne deaktivuje znak '-' v balicku babel
%ZADANIE EN
\includepdf[pages=-]{frontmatter/assignment.pdf}
%ZADANIE SK
\includepdf[pages=-]{frontmatter/zadanie.pdf}
\shorthandon{-}

%PODAKOVANIE
\input{frontmatter/acknowledgement}

%ABSTRAKT EN
\chapter*{Abstract}
The CSS is used in many web projects and the Sass is its extension. The Sass helps the developers to maintain the code readable even with large projects. The goal of the thesis is to write a tutorial for the Sass. All topics have at least one simple example to show the code in use. The second chapter is about the Compass framework, that brings more flexible way for work on CSS projects.  \\ \\
\textbf{\textsc{Keywords:}} sass, compass, css, tutorial



%ABSTRAKT SK
\selectlanguage{slovak}
\chapter*{Abstrakt}
S CSS sa stretávame pri väčšine webových projektov. Sass je rozšírením CSS o nové funkcie, ktoré pomáhajú sprehľadniť a zrýchliť vývoj v CSS. Zámerom práce je zoznámiť čitateľa so syntaxou a možnosťami preprocesora Sass. Jednotlivé funkcie sú s jednoduchými príkladmi na rýchlejšie pochopenie ich funkčnosti. Po prejdení základov Sass sa presúva pozornosť na knižnicu Compass, ktorá posúva prácu s projektmi písanými v Sass na ďalšiu úroveň.  \\ \\
\textbf{\textsc{Kľúčové slová:}} sass, compass, css, tutoriál


%PREDHOVOR
%\selectlanguage{english}
%\input{frontmatter/preamble}

%OBSAH

\phantomsection\label{index::doc}

%ZOZNAM ILUSTRACII
% \selectlanguage{english}
% \listoffigures

%ZOZNAM TABULIEK
\selectlanguage{english}
% \listoftables
% \thispagestyle{empty}

\newpage
\chapter*{Contents}
\input main.toc
\tableofcontents
\setcounter{page}{0}
\pagenumbering{arabic}






\chapter*{Introduction}
      \addcontentsline{toc}{chapter}{Introduction}
      \paragraph{}
      The look of the Internet had changed dramatically through last decades of existence. What was at begin just simple text file is today much more. The content is not the only important thing. The way how we are presented on the Internet has come to the front. For purpose of easier styling our web documents was released Cascading Style Sheets (CSS) style sheet language. In last years it was improved a lot and the number of features which was added to it is in hundreds. But there are always things that developers wants today, but it takes some time for approve appending new ones. And there is created space for another developers that come with their own solutions. One of them is CSS preprocessors.

      \paragraph{}
      The goal of the preprocessors is to add new features to existing styling language without loosing support of Internet browsers. The solution for that is to write converters that translate code written with syntax of our chosen preprocessor and translate it to the CSS. The output than could be optimized for example to save the time of page loading. There are more ways how this converting can be achieved. LESS is preferring javascript file which translates LESS on the client side. In other view, \href{http://sass-lang.com}{SASS} \cite{homepage:sass}, about which is this work, converts sass or scss syntax on the developer computer, or there exists frameworks like Ruby on Rails which has SASS converter integrated.

\chapter{Sass}


\section{First steps to the Sass}
\label{src/sass:first-steps-to-the-sass}

\subsection{History}
\label{src/sass:history}
Sass (Syntaticly awesome stylesheet) is meta-language created on top of CSS. It's main purpose is to provide more enhanced features to the CSS that are useful for creating manageable stylesheets. It was originally created by \href{http://www.hamptoncatlin.com/}{Hampton Catling} (http://www.hamptoncatlin.com/). He and \href{http://nex-3.com/}{Nathan Weizenbaum} (http://nex-3.com/) designed Sass through 2.0 version. Nathan is the primary designer of Sass and the main developer since its inception. In late 2008 joined the Sass team \href{http://chriseppstein.github.com/}{Chris Eppstein} (http://chriseppstein.github.com/). Chris and Nathan designed Sass from version 2.2. Another accomplishment of Chris is the \href{http://compass-style.org/}{Compass} (http://compass-style.org/), the first Sass-based framework. I am going to talk about it in next chapters, too.

The official implementation is done in Ruby. And through this manual I will be using only this one. There are attempts to make Sass interpreter in Javascript so you can run it on the server with Node.js or PHP version named \href{http://www.phpsass.com/}{PHPSass} (http://www.phpsass.com/). Sass is available under the \href{http://sass-lang.com/docs/yardoc/file.MIT-LICENSE.html}{MIT License} (http://sass-lang.com/docs/yardoc/file.MIT-LICENSE.html).


\subsection{Installation}
\label{src/sass:installation}
Now after few words from history is time to move on. Before we can start with the syntax, it's necessary to install the Sass interpreter. How I said before we will be using Ruby implementation. And because we have many operating systems with different dependencies I will write give you hints how to prepare your system.


\subsubsection{Windows}
\label{src/sass:windows}
The Windows does not come with  Ruby installed at default. The fastest way how to install it is to download \href{http://rubyinstaller.org/downloads/}{RubyInstaller} (http://rubyinstaller.org/downloads/).
When it is done, go to Start Menu, Accessories and run Command Prompt. But faster would be to press \emph{Win+R} and put in the \emph{cmd} command and press Run. Next step is same for all platforms.


\subsubsection{Linux}
\label{src/sass:linux}
This category is more general, while there are many distribution with Linux kernel. But the most used systems today have roots in Debian. In this group you can add Debian, Ubuntu, Linux Mint and so on. If you are using Arch, then I don't think think you need to read how to install Ruby.

For Debian based systems you just need to start Terminal and run in it

\begin{Verbatim}[commandchars=\\\{\}]
sudo apt-get install ruby1.9.1
\end{Verbatim}

It will ask your password and after that it will install Ruby.


\subsubsection{MacOS X}
\label{src/sass:macos-x}
\begin{notice}{note}{Todo}

This text possibly will be rewritten cause I figure out that on Mac is old version of Ruby.
\end{notice}

Your new Mac comes with Ruby installed. So only thing that you must do is to find your terminal. You can use Spotlight and write in it Terminal. And run it.


\subsubsection{Last step}
\label{src/sass:last-step}
At the end to install the Sass gem just write following command into the terminal or command prompt. When the installation ends, you are done.

\begin{Verbatim}[commandchars=\\\{\}]
gem install sass
\end{Verbatim}


\subsection{Hello World example}
\label{src/sass:hello-world-example}
It's good idea to try if it works. Run in terminal:

\begin{Verbatim}[commandchars=\\\{\}]
\PYG{n}{sass} \PYG{o}{\PYGZhy{}}\PYG{o}{\PYGZhy{}}\PYG{n}{scss}
\end{Verbatim}

Your input would be

\begin{Verbatim}[commandchars=\\\{\}]
\PYG{n+nv}{\PYGZdl{}header\PYGZhy{}color}\PYG{o}{:} \PYG{l+m+mh}{\PYGZsh{}fe3242}\PYG{p}{;}

\PYG{n+nt}{h1} \PYG{p}{\PYGZob{}}
    \PYG{n+na}{color}\PYG{o}{:} \PYG{n+nv}{\PYGZdl{}header\PYGZhy{}color}\PYG{p}{;}
\PYG{p}{\PYGZcb{}}
\end{Verbatim}

And when you are done press \emph{Ctrl+D}. You should get

\begin{Verbatim}[commandchars=\\\{\}]
\PYG{n+nt}{h1} \PYG{p}{\PYGZob{}}
    \PYG{k}{color}\PYG{o}{:} \PYG{l+m}{\PYGZsh{}fe3242}\PYG{p}{;}
\PYG{p}{\PYGZcb{}}
\end{Verbatim}

How can you see, the line starting with dollar sign disappeared and the color value has changed to value defined for \emph{\$header-color}. This is simple example of using variables in Sass. When there was no problems you can move to the next chapter.


\section{Sass Syntax}
\label{src/sass:sass-syntax}
After details how to setup up your working environment for Sass, it's time to move on to the syntax of this language. How I said earlier anything written in CSS is valid Sass code. It's not always true. The thing is that Sass has two possible syntaxes. The older one, called simply Sass with extension \emph{.sass}, takes inspiration in \href{http://haml.info/}{Haml} (http://haml.info/). There are no semicolons, no curly brackets and few more differences from style that will be used. The important thing in it is using indention. If you met with languages like Ruby or Python, than you understand. For people who have no clue about what I'm talking is here small example.

First is the code in Sass (.sass extension)

\begin{Verbatim}[commandchars=\\\{\}]
\PYG{n+na}{\PYGZsh{}main}
    \PYG{n+na}{background}\PYG{o}{:} \PYG{n+nb}{red}\PYG{p}{;}
    \PYG{n+na}{color}\PYG{o}{:} \PYG{n+nb}{white}
    \PYG{n}{a}
        \PYG{n+no}{font}\PYG{o}{:}
            \PYG{n}{weight}\PYG{o}{:} \PYG{n+no}{bold}
            \PYG{n+no}{size}\PYG{o}{:} \PYG{l+m+mi}{2}\PYG{k+kt}{em}
            \PYG{n}{family}\PYG{o}{:} \PYG{n+no}{serif}
        \PYG{n+no}{color}\PYG{o}{:} \PYG{n+nb}{yellow}
        \PYG{o}{\PYGZam{}:}\PYG{n}{hover}
            \PYG{n+no}{color}\PYG{o}{:} \PYG{n+nb}{green}
\end{Verbatim}

And now CSS equivalent to code above.

\begin{Verbatim}[commandchars=\\\{\}]
\PYG{n+nf}{\PYGZsh{}main} \PYG{p}{\PYGZob{}}
    \PYG{k}{background}\PYG{o}{:} \PYG{n+nb}{red}\PYG{p}{;}
    \PYG{k}{color}\PYG{o}{:} \PYG{n+nb}{white}\PYG{p}{;}
\PYG{p}{\PYGZcb{}}
\PYG{n+nf}{\PYGZsh{}main} \PYG{n+nt}{a} \PYG{p}{\PYGZob{}}
    \PYG{k}{font\PYGZhy{}weight}\PYG{o}{:} \PYG{k}{bold}\PYG{p}{;}
    \PYG{k}{font\PYGZhy{}size}\PYG{o}{:} \PYG{l+m}{2em}\PYG{p}{;}
    \PYG{k}{font\PYGZhy{}family}\PYG{o}{:} \PYG{k}{serif}\PYG{p}{;}
    \PYG{k}{color}\PYG{o}{:} \PYG{n+nb}{yellow}\PYG{p}{;}
\PYG{p}{\PYGZcb{}}
\PYG{n+nf}{\PYGZsh{}main} \PYG{n+nt}{a}\PYG{n+nd}{:hover} \PYG{p}{\PYGZob{}}
    \PYG{k}{color}\PYG{o}{:} \PYG{n+nb}{green}\PYG{p}{;}
\PYG{p}{\PYGZcb{}}
\end{Verbatim}

This approach to the syntax has some advantages and if you have some experience with languages where indention is so important than go for it. But in this materials I will be using most often the SCSS (Sassy CSS) syntax. It's more similar to CSS so there would be no problems to start using Sass, what is main purpose of this tutorial.


\subsection{Variables}
\label{src/sass:variables}
How often happened to you that you were writing CSS, in which you need to set up color for some element, but you don't remember the code of used color? You can still find it in document, but it could take some time.
Or another example. You got some code at which had worked some other developer and only thing that you need to do is to change colors of all links in the document. The problem is that you don't know in how many declarations is that colored used in document and how we can see later, code written in Sass is often spited in many files. If the previous developer used variable to store the color value, than your work would be just to find the declaration of color for link and change it.

In this example situations variables come to be handy. It's true that they are often used as constants in Sass. But there are no problems to change their values later if it needed. But it is no good practice to do so. While than it can start to be mess and you can not be sure which value is used at the moment so easily. The definition of variable starts with symbol \emph{\$} following with the variable name, double-colon and the variable value. The value can be color code in any format supported in CSS, string, number or length with unit.

\begin{Verbatim}[commandchars=\\\{\}]
\PYG{n+nv}{\PYGZdl{}color\PYGZhy{}var\PYGZhy{}name}\PYG{o}{:} \PYG{n+nf}{rgba}\PYG{p}{(}\PYG{l+m+mi}{42}\PYG{o}{,}\PYG{l+m+mi}{42}\PYG{o}{,}\PYG{l+m+mi}{42}\PYG{o}{,}\PYG{l+m+mi}{1}\PYG{p}{)}\PYG{p}{;}
\PYG{n+nv}{\PYGZdl{}length\PYGZhy{}var\PYGZhy{}name}\PYG{o}{:} \PYG{l+m+mi}{960}\PYG{k+kt}{px}\PYG{p}{;}
\PYG{n+nv}{\PYGZdl{}string\PYGZhy{}var\PYGZhy{}name}\PYG{o}{:} \PYG{l+s+s2}{\PYGZdq{}}\PYG{l+s+s2}{\textbar{}}\PYG{l+s+s2}{\PYGZdq{}}\PYG{p}{;}
\PYG{n+nv}{\PYGZdl{}number}\PYG{o}{:} \PYG{l+m+mi}{0}\PYG{l+m+mf}{.2}\PYG{p}{;}

\PYG{n+nn}{\PYGZsh{}}\PYG{n+nn}{main} \PYG{p}{\PYGZob{}}
    \PYG{n+na}{width}\PYG{o}{:} \PYG{n+nv}{\PYGZdl{}length\PYGZhy{}var\PYGZhy{}name}
\PYG{p}{\PYGZcb{}}
\PYG{n+nt}{a} \PYG{p}{\PYGZob{}}
    \PYG{n+na}{color}\PYG{o}{:} \PYG{n+nv}{\PYGZdl{}color\PYGZhy{}var\PYGZhy{}name}\PYG{p}{;}
    \PYG{n+na}{opacity}\PYG{o}{:} \PYG{n+nv}{\PYGZdl{}number}\PYG{p}{;}
\PYG{p}{\PYGZcb{}}
\end{Verbatim}

\begin{Verbatim}[commandchars=\\\{\}]
\PYG{n+nf}{\PYGZsh{}main} \PYG{p}{\PYGZob{}}
    \PYG{k}{width}\PYG{o}{:} \PYG{l+m}{960px}\PYG{p}{;}
\PYG{p}{\PYGZcb{}}
\PYG{n+nt}{a} \PYG{p}{\PYGZob{}}
    \PYG{k}{color}\PYG{o}{:} \PYG{n}{rgba}\PYG{p}{(}\PYG{l+m}{42}\PYG{o}{,}\PYG{l+m}{42}\PYG{o}{,}\PYG{l+m}{42}\PYG{o}{,}\PYG{l+m}{1}\PYG{p}{);}
    \PYG{k}{opacity}\PYG{o}{:} \PYG{l+m}{0}\PYG{o}{.}\PYG{l+m}{2}\PYG{p}{;}
\PYG{p}{\PYGZcb{}}
\end{Verbatim}

\begin{notice}{note}{Note:}
\textbf{Naming conventions}
They are inherited from CSS. The name for variable should be created from alphanumeric symbols and separated by hyphen. The name should say enough about the value that is saved in it. Try to avoid names like \emph{\$red-color} and than use it for all your links. Better approach is to create some color scheme like \emph{\$red: \#E03838;},than create \emph{\$link-color: \$red;} and use it for links. If you came to state that you need to change the color from red to green, you will just declare \emph{\$green} and set the \emph{\$link-color} to it. It's better, because if you stay with \emph{\$red-color}, than you will probably change the value stored in \emph{\$red-color} to green and it does not make sense.
\end{notice}


\subsection{Nesting}
\label{src/sass:nesting}
I'd like to start with simple CSS code for horizontal navigation.

\begin{Verbatim}[commandchars=\\\{\}]
\PYG{n+nt}{nav} \PYG{p}{\PYGZob{}} \PYG{k}{position}\PYG{o}{:} \PYG{k}{absolute}\PYG{p}{;} \PYG{k}{right}\PYG{o}{:} \PYG{l+m}{5em}\PYG{p}{;} \PYG{k}{bottom}\PYG{o}{:} \PYG{l+m}{2em}\PYG{p}{;} \PYG{p}{\PYGZcb{}}
\PYG{n+nt}{nav} \PYG{n+nt}{ul} \PYG{p}{\PYGZob{}} \PYG{k}{list\PYGZhy{}style}\PYG{o}{:} \PYG{k}{none}\PYG{p}{;} \PYG{p}{\PYGZcb{}}
\PYG{n+nt}{nav} \PYG{n+nt}{ul} \PYG{n+nt}{li} \PYG{p}{\PYGZob{}} \PYG{k}{display}\PYG{o}{:} \PYG{k}{inline}\PYG{p}{;} \PYG{p}{\PYGZcb{}}
\PYG{n+nt}{nav} \PYG{n+nt}{ul} \PYG{n+nt}{li} \PYG{n+nt}{a} \PYG{p}{\PYGZob{}} \PYG{k}{color}\PYG{o}{:} \PYG{l+m}{\PYGZsh{}4590DE}\PYG{p}{;} \PYG{k}{text\PYGZhy{}decoration}\PYG{o}{:} \PYG{k}{none}\PYG{p}{;} \PYG{p}{\PYGZcb{}}
\PYG{n+nt}{nav} \PYG{n+nt}{ul} \PYG{n+nt}{li} \PYG{n+nt}{a}\PYG{n+nd}{:hover} \PYG{p}{\PYGZob{}} \PYG{k}{text\PYGZhy{}decoration}\PYG{o}{:} \PYG{k}{underline}\PYG{p}{;} \PYG{p}{\PYGZcb{}}
\end{Verbatim}

You probably met with similar code. If you look at it you can see that I repeated some selectors. In final style they are important, but when you are writing code, you don't want to repeat yourself. Computers are good for repetitive work so why don't use them for this too? Sass has solution for this. It's called nesting. And it's main idea is that child elements are written inside of the parent curly brackets. So than I can rewrite the CSS code into.

\begin{Verbatim}[commandchars=\\\{\}]
\PYG{n+nt}{nav} \PYG{p}{\PYGZob{}}
    \PYG{n+na}{position}\PYG{o}{:} \PYG{n+no}{absolute}\PYG{p}{;}
    \PYG{n+na}{right}\PYG{o}{:} \PYG{l+m+mi}{5}\PYG{k+kt}{em}\PYG{p}{;}
    \PYG{n+na}{bottom}\PYG{o}{:} \PYG{l+m+mi}{2}\PYG{k+kt}{em}\PYG{p}{;}
    \PYG{n+nt}{ul} \PYG{p}{\PYGZob{}}
        \PYG{n+na}{list\PYGZhy{}style}\PYG{o}{:} \PYG{n+no}{none}\PYG{p}{;}
        \PYG{n+nt}{li} \PYG{p}{\PYGZob{}}
            \PYG{n+na}{display}\PYG{o}{:} \PYG{n+no}{inline}\PYG{p}{;}
            \PYG{n+nt}{a} \PYG{p}{\PYGZob{}}
                \PYG{n+na}{color}\PYG{o}{:} \PYG{l+m+mh}{\PYGZsh{}4590DE}\PYG{p}{;}
                \PYG{n+na}{text\PYGZhy{}decoration}\PYG{o}{:} \PYG{n+no}{none}\PYG{p}{;}
                \PYG{k}{\PYGZam{}}\PYG{n+nd}{:}\PYG{n+nd}{hover} \PYG{p}{\PYGZob{}}
                    \PYG{n+na}{text\PYGZhy{}decoration}\PYG{o}{:} \PYG{n+no}{underline}\PYG{p}{;}
                \PYG{p}{\PYGZcb{}}
            \PYG{p}{\PYGZcb{}}
        \PYG{p}{\PYGZcb{}}
    \PYG{p}{\PYGZcb{}}
\PYG{p}{\PYGZcb{}}
\end{Verbatim}

The indention in code is not important, but it's recommended, for easier reading of the code. How you can see I didn't repeat any selector. And there is interesting syntax with the ampersand. \emph{\&:hover}. The ampersand stands for the parent selector. The reason why I didn't used it for previous declaration is that it's added there automatically. So if you write

\begin{Verbatim}[commandchars=\\\{\}]
\PYG{n+nt}{nav} \PYG{p}{\PYGZob{}}
    \PYG{n+nt}{ul} \PYG{p}{\PYGZob{}}

    \PYG{p}{\PYGZcb{}}
\PYG{p}{\PYGZcb{}}
\end{Verbatim}

it can be rewritten using \emph{\&}-syntax to

\begin{Verbatim}[commandchars=\\\{\}]
\PYG{n+nt}{nav} \PYG{p}{\PYGZob{}}
    \PYG{k}{\PYGZam{}} \PYG{n+nt}{ul} \PYG{p}{\PYGZob{}}

    \PYG{p}{\PYGZcb{}}
\PYG{p}{\PYGZcb{}}
\end{Verbatim}

We need to refer on the parent selector in case that there is no need for space between selectors. For example when we use pseudo-selectors. Or if there is class which we want to style if it's for some specific selector.

\begin{Verbatim}[commandchars=\\\{\}]
\PYG{n+nt}{table} \PYG{p}{\PYGZob{}}
    \PYG{k}{\PYGZam{}}\PYG{n+nc}{.}\PYG{n+nc}{users\PYGZhy{}mode} \PYG{p}{\PYGZob{}}
    \PYG{p}{\PYGZcb{}}
\PYG{p}{\PYGZcb{}}
\end{Verbatim}

will be translated to

\begin{Verbatim}[commandchars=\\\{\}]
\PYG{n+nt}{table} \PYG{p}{\PYGZob{}} \PYG{p}{\PYGZcb{}}
\PYG{n+nt}{table}\PYG{n+nc}{.users\PYGZhy{}mode} \PYG{p}{\PYGZob{}}   \PYG{p}{\PYGZcb{}}
\end{Verbatim}

There is one more way where to use nesting. And it's for CSS properties. Some of them are created with some prefix like font-, text-, border-,etc. and if you are going to set more of them you can use the short version, but sometimes you need to specify it more explicit. And than comes nesting handy. The example would be best for it.

\begin{Verbatim}[commandchars=\\\{\}]
\PYG{n+nc}{.}\PYG{n+nc}{block} \PYG{p}{\PYGZob{}}
    \PYG{n+na}{border}\PYG{o}{:} \PYG{p}{\PYGZob{}}
        \PYG{n+na}{width}\PYG{o}{:} \PYG{n+no}{thin} \PYG{n+no}{thin} \PYG{l+m+mi}{0} \PYG{l+m+mi}{0}\PYG{p}{;}
        \PYG{n+na}{color}\PYG{o}{:} \PYG{n+nb}{red} \PYG{n+nb}{blue}\PYG{p}{;}
        \PYG{n+na}{style}\PYG{o}{:} \PYG{n+no}{solid}\PYG{p}{;}
    \PYG{p}{\PYGZcb{}}
\PYG{p}{\PYGZcb{}}
\end{Verbatim}

\begin{Verbatim}[commandchars=\\\{\}]
\PYG{n+nc}{.block} \PYG{p}{\PYGZob{}}
    \PYG{k}{border\PYGZhy{}width}\PYG{o}{:} \PYG{k}{thin} \PYG{k}{thin} \PYG{l+m}{0} \PYG{l+m}{0}\PYG{p}{;}
    \PYG{k}{border\PYGZhy{}color}\PYG{o}{:} \PYG{n+nb}{red} \PYG{n+nb}{blue}\PYG{p}{;}
    \PYG{k}{border\PYGZhy{}style}\PYG{o}{:} \PYG{k}{solid}\PYG{p}{;}
\PYG{p}{\PYGZcb{}}
\end{Verbatim}


\subsection{SassScript}
\label{src/sass:sassscript}
Extra features that you don't find in CSS brings the SassScript. It allows to use arithmetic operations, interpolation and some extra functions. If you want to just try some of it without writing files, than for you is there Interactive Shell.

\begin{Verbatim}[commandchars=\\\{\}]
\PYG{n+nt}{sass} \PYG{n+nt}{\PYGZhy{}i}
\PYG{o}{\PYGZgt{}}\PYG{o}{\PYGZgt{}} \PYG{n+nt}{1px} \PYG{o}{+} \PYG{n+nt}{1px} \PYG{o}{+} \PYG{n+nt}{1px}
\PYG{n+nt}{3px}
\PYG{o}{\PYGZgt{}}\PYG{o}{\PYGZgt{}} \PYG{n+nn}{\PYGZsh{}}\PYG{n+nn}{123} \PYG{n+nt}{\PYGZhy{}} \PYG{n+nn}{\PYGZsh{}}\PYG{n+nn}{010101}
\PYG{n+nn}{\PYGZsh{}}\PYG{n+nn}{122334}
\PYG{o}{\PYGZgt{}}\PYG{o}{\PYGZgt{}} \PYG{n+nn}{\PYGZsh{}}\PYG{n+nn}{777} \PYG{o}{+} \PYG{n+nn}{\PYGZsh{}}\PYG{n+nn}{888}
\PYG{n+nt}{white}
\end{Verbatim}

\textbf{Data types}

The SassScript supports 6 data types. There is no need to declare them. It will be automatically done. They are
\begin{itemize}
\item {} 
numbers, e.g. 1.2, 13, 10px

\item {} 
strings with and without quotes, e.g. ``cube'', `triangle', line

\item {} 
colors, e.g. red, \#123456, rgba(234,123,0, 0.8)

\item {} 
booleans, e.g. true, false

\item {} 
null

\item {} 
list of values separated by spaces or commas, e.g. thin solid black

\end{itemize}

You don't need thing about these types a lot. Only in cases that you store for example string into variable, but you want to use it to set a size of font. In this case it doesn't make sense.

\begin{notice}{note}{Todo}

rethink this section if it's important. many of these things are used if you write your own mixins.
\end{notice}


\subsection{@-rules and directives}
\label{src/sass:rules-and-directives}
Sass supports all CSS @-rules like \emph{@import}, \emph{@media} or \emph{@font-face}, but some of them extend and gives them more power.


\subsubsection{@import}
\label{src/sass:import}
With import rule you will meet often using Sass. It extends CSS import rule, so you can import \emph{.scss} and \emph{.sass} files. The output will be merged into one single CSS file and all variables and mixins defined in the imported files will be available in the main file. With this behavior you can split your styles into smaller files defining specific elements. It makes easy to append or edit the code.

There are some special circumstances at which will the @import rule works like the CSS.
\begin{itemize}
\item {} 
The file's extension is \emph{.css}.

\item {} 
The filename begins with \emph{http://}.

\item {} 
If the filename is \emph{url()}.

\item {} 
If the \emph{@import} has any media queries.

\end{itemize}

\begin{Verbatim}[commandchars=\\\{\}]
\PYG{k}{@import} \PYG{l+s+s2}{\PYGZdq{}}\PYG{l+s+s2}{cube.css}\PYG{l+s+s2}{\PYGZdq{}}\PYG{p}{;}
\PYG{k}{@import} \PYG{l+s+s2}{\PYGZdq{}}\PYG{l+s+s2}{cube}\PYG{l+s+s2}{\PYGZdq{}} \PYG{n}{screen}\PYG{p}{;}
\PYG{k}{@import} \PYG{l+s+s2}{\PYGZdq{}}\PYG{l+s+s2}{http://cube.edu/style}\PYG{l+s+s2}{\PYGZdq{}}\PYG{p}{;}
\PYG{k}{@import} \PYG{l+s+sx}{url(}\PYG{l+s+sx}{cube}\PYG{l+s+sx}{)}\PYG{p}{;}
\end{Verbatim}

\begin{Verbatim}[commandchars=\\\{\}]
\PYG{k}{@import} \PYG{l+s+s2}{\PYGZdq{}cube.css\PYGZdq{}}\PYG{p}{;}
\PYG{k}{@import} \PYG{l+s+s2}{\PYGZdq{}cube\PYGZdq{}} \PYG{n+nt}{screen}\PYG{p}{;}
\PYG{k}{@import} \PYG{l+s+s2}{\PYGZdq{}http://cube.edu/style\PYGZdq{}}\PYG{p}{;}
\PYG{k}{@import} \PYG{n+nt}{url}\PYG{o}{(}\PYG{n+nt}{cube}\PYG{o}{)}\PYG{p}{;}
\end{Verbatim}

If we want to import the file \emph{cube.scss} we can write:

\begin{Verbatim}[commandchars=\\\{\}]
@import "cube.scss";
\end{Verbatim}

or just simply:

\begin{Verbatim}[commandchars=\\\{\}]
{}`@import "cube";{}`
\end{Verbatim}

If you want to import more files, it's possible to write:

\begin{Verbatim}[commandchars=\\\{\}]
@import "first", "second";
\end{Verbatim}

If you name the \emph{.scss} file with underscore before filename \emph{\_cube.scss}, than it's code will be added to the main \emph{.css} file, but it will be not compiled to CSS at own. In \emph{@import} you don't need to write the underscore, but it's important that in the same folder can not be  more files with the same name. (If in folder is \emph{cube.scss}, than you can not use \emph{\_cube.scss}). This type of naming of files is called partials.

One of earlier mentioned features of Sass is nesting and it's possible to use it with \emph{@import}.  Most of time will you use the \emph{@import} at the top of the document. But there can come situation when it would be handy to include whole another file into some class. At that case you can call \emph{@import} under class.

The best way to understand is through example. The \emph{box.scss} and \emph{screen.scss} contain following code

\begin{Verbatim}[commandchars=\\\{\}]
\PYG{c+c1}{// Content of the box.scss}
\PYG{n+nc}{.}\PYG{n+nc}{box} \PYG{p}{\PYGZob{}}
    \PYG{n+na}{color}\PYG{o}{:} \PYG{n+nb}{red}\PYG{p}{;}
    \PYG{n+nc}{.}\PYG{n+nc}{button} \PYG{p}{\PYGZob{}}
        \PYG{n+na}{background}\PYG{o}{:} \PYG{l+m+mh}{\PYGZsh{}444}\PYG{p}{;}
    \PYG{p}{\PYGZcb{}}
\PYG{p}{\PYGZcb{}}

\PYG{c+c1}{// Content of the screen.scss}
\PYG{n+nc}{.}\PYG{n+nc}{screen} \PYG{p}{\PYGZob{}}
    \PYG{k}{@import} \PYG{l+s+s1}{\PYGZsq{}}\PYG{l+s+s2}{box.scss}\PYG{l+s+s2}{\PYGZsq{}}\PYG{p}{;}
\PYG{p}{\PYGZcb{}}
\end{Verbatim}

The compiled version is

\begin{Verbatim}[commandchars=\\\{\}]
\PYG{n+nc}{.screen} \PYG{n+nc}{.box} \PYG{p}{\PYGZob{}}
    \PYG{k}{color}\PYG{o}{:} \PYG{n+nb}{red}\PYG{p}{;}
\PYG{p}{\PYGZcb{}}
\PYG{n+nc}{.screen} \PYG{n+nc}{.box} \PYG{n+nc}{.button} \PYG{p}{\PYGZob{}}
    \PYG{k}{background}\PYG{o}{:} \PYG{l+m}{\PYGZsh{}444}\PYG{p}{;}
\PYG{p}{\PYGZcb{}}
\end{Verbatim}

There are few exceptions. There exists directives that can be only at the base level of the document. So if you are calling \emph{@import} into selector than the imported file can not contain \emph{@mixin} and \emph{@charset}. It's not possible to @import in mixins and control directives.


\subsubsection{@media}
\label{src/sass:media}
\emph{@media} directive can be used as defined in the plain CSS, but it has one extra capability - it can be nested in CSS rule.  If it appears nested, than it bubble to the base level, containing all selectors in which it's included. This approach helps to make your code readable if you are using the @media.

\begin{Verbatim}[commandchars=\\\{\}]
\PYG{n+nc}{.}\PYG{n+nc}{sidebar} \PYG{p}{\PYGZob{}}
    \PYG{n+na}{width}\PYG{o}{:} \PYG{l+m+mi}{300}\PYG{k+kt}{px}\PYG{p}{;}
    \PYG{k}{@media} \PYG{n+nt}{screen} \PYG{n+nt}{and} \PYG{o}{(}\PYG{n+nt}{orientation}\PYG{n+nd}{:} \PYG{n+nt}{landscape}\PYG{o}{)} \PYG{p}{\PYGZob{}}
        \PYG{n+na}{width}\PYG{o}{:} \PYG{l+m+mi}{500}\PYG{k+kt}{px}\PYG{p}{;}
    \PYG{p}{\PYGZcb{}}
\PYG{p}{\PYGZcb{}}
\end{Verbatim}

\begin{Verbatim}[commandchars=\\\{\}]
\PYG{n+nc}{.sidebar} \PYG{p}{\PYGZob{}}
    \PYG{k}{width}\PYG{o}{:} \PYG{l+m}{300px}\PYG{p}{;}
\PYG{p}{\PYGZcb{}}

\PYG{k}{@media} \PYG{n+nt}{screen} \PYG{n+nt}{and} \PYG{o}{(}\PYG{n+nt}{orientation}\PYG{o}{:} \PYG{n+nt}{landscape}\PYG{o}{)} \PYG{p}{\PYGZob{}}
    \PYG{n+nc}{.sidebar} \PYG{p}{\PYGZob{}}
        \PYG{k}{width}\PYG{o}{:} \PYG{l+m}{500px}\PYG{p}{;}
    \PYG{p}{\PYGZcb{}}
\PYG{p}{\PYGZcb{}}
\end{Verbatim}

This way you don't break the flow of your selectors that are nested into each and again you don't need to repeat to writing the selector that you want to specify with @media. You can complain that than you often write the characteristic for the \emph{@media}, but we have variables. This is nice place where you can use it.

\begin{Verbatim}[commandchars=\\\{\}]
\$landscape: 'screen and (orientation: landscape)';

.sidebar \PYGZob{}
    width: 300px;
    @media \#\PYGZob{}\$landscape\PYGZcb{} \PYGZob{}
        width: 500px;
    \PYGZcb{}
\PYGZcb{}

.content \PYGZob{}
    width: 400px;
    @media \#\PYGZob{}\$landscape\PYGZcb{} \PYGZob{}
        width: 600px;
    \PYGZcb{}
\PYGZcb{}
\end{Verbatim}

\begin{Verbatim}[commandchars=\\\{\}]
\PYG{k}{@media} \PYG{n+nt}{screen} \PYG{n+nt}{and} \PYG{o}{(}\PYG{n+nt}{orientation}\PYG{o}{:} \PYG{n+nt}{landscape}\PYG{o}{)} \PYG{p}{\PYGZob{}}
    \PYG{n+nc}{.sidebar} \PYG{p}{\PYGZob{}}
        \PYG{k}{width}\PYG{o}{:} \PYG{l+m}{500px}\PYG{p}{;}
    \PYG{p}{\PYGZcb{}}
    \PYG{n+nc}{.content} \PYG{p}{\PYGZob{}}
        \PYG{k}{width}\PYG{o}{:} \PYG{l+m}{600px}\PYG{p}{;}
    \PYG{p}{\PYGZcb{}}
\PYG{p}{\PYGZcb{}}

\PYG{n+nc}{.sidebar} \PYG{p}{\PYGZob{}}
    \PYG{k}{width}\PYG{o}{:} \PYG{l+m}{300px}\PYG{p}{;}
\PYG{p}{\PYGZcb{}}
\PYG{n+nc}{.content} \PYG{p}{\PYGZob{}}
    \PYG{k}{width}\PYG{o}{:} \PYG{l+m}{400px}\PYG{p}{;}
\PYG{p}{\PYGZcb{}}
\end{Verbatim}

You can define  more @media properties for specific devices at the start and if you need to change some properties you don't need to look through documents where you write device-specific rules and no selectors were written twice.


\subsubsection{@extend}
\label{src/sass:extend}
There are often cases when you need to use the all rules from one selector and add only some new. Most used way how to do that is using some general class and than more specific class that sets the different properties. Than the HTML will be following

\begin{Verbatim}[commandchars=\\\{\}]
\PYG{n+nt}{\PYGZlt{}div} \PYG{n+na}{class=}\PYG{l+s}{\PYGZdq{}error error\PYGZhy{}login\PYGZdq{}}\PYG{n+nt}{\PYGZgt{}}
    Sorry, bad login or password. Try it again.
\PYG{n+nt}{\PYGZlt{}/div\PYGZgt{}}
\end{Verbatim}

The css to the code will be

\begin{Verbatim}[commandchars=\\\{\}]
\PYG{n+nc}{.error} \PYG{p}{\PYGZob{}}
    \PYG{k}{border}\PYG{o}{:} \PYG{k}{thin} \PYG{k}{solid} \PYG{l+m}{\PYGZsh{}FF5151}\PYG{p}{;}
    \PYG{k}{background\PYGZhy{}color}\PYG{o}{:} \PYG{l+m}{\PYGZsh{}F9E9E9}\PYG{p}{;}
\PYG{p}{\PYGZcb{}}

\PYG{n+nc}{.error\PYGZhy{}login} \PYG{p}{\PYGZob{}}
    \PYG{k}{border\PYGZhy{}width}\PYG{o}{:} \PYG{k}{thick}\PYG{p}{;}
\PYG{p}{\PYGZcb{}}
\end{Verbatim}

This method is functional and it's often used, but you must not forgot the error class. The \emph{@extend} directive helps to avoid the some problems that are possible using this way. Than in the HTML will be written

\begin{Verbatim}[commandchars=\\\{\}]
\PYG{n+nt}{\PYGZlt{}div} \PYG{n+na}{class=}\PYG{l+s}{\PYGZdq{}error\PYGZhy{}login\PYGZdq{}}\PYG{n+nt}{\PYGZgt{}}
    Sorry, bad login or password. Try it again.
\PYG{n+nt}{\PYGZlt{}/div\PYGZgt{}}
\end{Verbatim}

\begin{Verbatim}[commandchars=\\\{\}]
\PYG{n+nc}{.}\PYG{n+nc}{error} \PYG{p}{\PYGZob{}}
    \PYG{n+na}{border}\PYG{o}{:} \PYG{n+no}{thin} \PYG{n+no}{solid} \PYG{l+m+mh}{\PYGZsh{}FF5151}\PYG{p}{;}
    \PYG{n+na}{background\PYGZhy{}color}\PYG{o}{:} \PYG{l+m+mh}{\PYGZsh{}F9E9E9}\PYG{p}{;}
\PYG{p}{\PYGZcb{}}
\PYG{n+nc}{.}\PYG{n+nc}{error\PYGZhy{}login} \PYG{p}{\PYGZob{}}
    \PYG{k}{@extend} \PYG{n+nc}{.}\PYG{n+nc}{error}\PYG{o}{;}
    \PYG{n+nt}{border\PYGZhy{}width}\PYG{n+nd}{:} \PYG{n+nt}{thick}\PYG{o}{;}
\PYG{p}{\PYGZcb{}}
\end{Verbatim}

\begin{Verbatim}[commandchars=\\\{\}]
\PYG{n+nc}{.error}\PYG{o}{,} \PYG{n+nc}{.error\PYGZhy{}login} \PYG{p}{\PYGZob{}}
    \PYG{k}{border}\PYG{o}{:} \PYG{k}{thin} \PYG{k}{solid} \PYG{l+m}{\PYGZsh{}FF5151}\PYG{p}{;}
    \PYG{k}{background\PYGZhy{}color}\PYG{o}{:} \PYG{l+m}{\PYGZsh{}F9E9E9}\PYG{p}{;}
\PYG{p}{\PYGZcb{}}
\PYG{n+nc}{.error\PYGZhy{}login} \PYG{p}{\PYGZob{}}
    \PYG{k}{border\PYGZhy{}width}\PYG{o}{:} \PYG{k}{thick}\PYG{p}{;}
\PYG{p}{\PYGZcb{}}
\end{Verbatim}

\emph{@extend} works by inserting extending selector anywhere the extended selector appears. For better illustration I add example.

\begin{Verbatim}[commandchars=\\\{\}]
\PYG{n+nc}{.}\PYG{n+nc}{error} \PYG{p}{\PYGZob{}}
    \PYG{n+na}{border}\PYG{o}{:} \PYG{n+no}{thin} \PYG{n+no}{solid} \PYG{n+nb}{red}\PYG{p}{;}
    \PYG{n+na}{padding}\PYG{o}{:} \PYG{l+m+mf}{.5}\PYG{k+kt}{em}\PYG{p}{;}
    \PYG{n+na}{color}\PYG{o}{:} \PYG{n+nb}{red}\PYG{p}{;}
\PYG{p}{\PYGZcb{}}
\PYG{n+nc}{.}\PYG{n+nc}{error}\PYG{n+nc}{.}\PYG{n+nc}{icon} \PYG{p}{\PYGZob{}}
    \PYG{n+na}{background}\PYG{o}{:} \PYG{l+s+sx}{url(}\PYG{l+s+sx}{\PYGZsq{}images/error.png\PYGZsq{}}\PYG{l+s+sx}{)}\PYG{p}{;}
\PYG{p}{\PYGZcb{}}
\PYG{n+nc}{.}\PYG{n+nc}{error\PYGZhy{}login} \PYG{p}{\PYGZob{}}
    \PYG{k}{@expand} \PYG{n+nc}{.}\PYG{n+nc}{error}\PYG{o}{;}
    \PYG{n+nt}{font\PYGZhy{}weight}\PYG{n+nd}{:} \PYG{n+nt}{bold}\PYG{o}{;}
\PYG{p}{\PYGZcb{}}
\end{Verbatim}

\begin{Verbatim}[commandchars=\\\{\}]
\PYG{n+nc}{.error}\PYG{o}{,} \PYG{n+nc}{.error\PYGZhy{}login} \PYG{p}{\PYGZob{}}
    \PYG{k}{border}\PYG{o}{:} \PYG{k}{thin} \PYG{k}{solid} \PYG{n+nb}{red}\PYG{p}{;}
    \PYG{k}{padding}\PYG{o}{:} \PYG{l+m}{.5em}\PYG{p}{;}
    \PYG{k}{color}\PYG{o}{:} \PYG{n+nb}{red}\PYG{p}{;}
\PYG{p}{\PYGZcb{}}

\PYG{n+nc}{.error}\PYG{n+nc}{.icon}\PYG{o}{,} \PYG{n+nc}{.error\PYGZhy{}login}\PYG{n+nc}{.icon} \PYG{p}{\PYGZob{}}
    \PYG{k}{background}\PYG{o}{:} \PYG{l+s+sx}{url(\PYGZsq{}images/error.png\PYGZsq{})}\PYG{p}{;}
\PYG{p}{\PYGZcb{}}

\PYG{n+nc}{.error\PYGZhy{}login} \PYG{p}{\PYGZob{}}
    \PYG{k}{font\PYGZhy{}weight}\PYG{o}{:} \PYG{k}{bold}\PYG{p}{;}
\PYG{p}{\PYGZcb{}}
\end{Verbatim}

If you are familiar with Object Oriented languages as Java or C++ you sure know the meaning of abstract class or function. There are not allowed instances from abstract classes, so they must be inherited by another class. And in Sass exists similar way how to define the selector. It can be called ``placeholder selectors''. They are defined in Sass version of code, but they are not compiled to the CSS. Only if they are extended by another selector. It helps to avoid names collisions and the in the output CSS they show up only if they are needed. They are most of time used if you are creating framework. The syntax difference from the selectors for classes and ids only in first characters. You don't use the \emph{.} or \emph{\#}, but \emph{\%}. So ``placeholder selector'' for the error could be \emph{\%error}. Everything else works like it is normal selector.

The main difference between using extend and mixins is in the output CSS. Say that we want to create four buttons and only the color of the background would be changed. If you use for that mixins the output would have the same code for every button generated, and only the color codes would be different. On other hand, if you extend generic ``placeholder selector''  for buttons and set for each one only different color, than the generated CSS will have few lines less code. In situation when you want to load your site as fast as possible is this approach good idea. But always this things depends on the situation.

If you want to use \emph{@extend} inside of the media block, there is some restrictions. You can extend only selectors that are inside of the media block.

\begin{notice}{note}{Todo}

at this point I need to make some example files and test how it would behave.
\end{notice}


\subsection{Mixins}
\label{src/sass:mixins}
Some of many advantages of Sass is keeping your code readable and don't repeating yourself. For the DRY (don't repeat yourself) exists mixins. If you work with programming languages before, you can say that they look like functions. But like many things in Sass comes from Ruby universe, mixins exists there too. The way how mixins work is to include their code at the place where they were called.

\textbf{Defining a Mixin with {}`@mixin{}`}

Mixins are defined with the directive \emph{@mixin} following with the name of mixin and optionally the arguments. After that is there a block containing content of mixin closed into curly brackets.

\begin{Verbatim}[commandchars=\\\{\}]
\PYG{k}{@mixin}\PYG{n+nf}{ button} \PYG{p}{\PYGZob{}}
    \PYG{n+na}{border}\PYG{o}{:} \PYG{n+no}{thin} \PYG{n+no}{solid} \PYG{l+m+mh}{\PYGZsh{}40AECA}\PYG{p}{;}
    \PYG{n+na}{background}\PYG{o}{:} \PYG{l+m+mh}{\PYGZsh{}85C7D8}\PYG{p}{;}
    \PYG{n+na}{border\PYGZhy{}radius}\PYG{o}{:} \PYG{l+m+mi}{5}\PYG{k+kt}{px}\PYG{p}{;}
    \PYG{n+na}{color}\PYG{o}{:} \PYG{n+nb}{white}\PYG{p}{;}
    \PYG{k}{\PYGZam{}}\PYG{n+nd}{:}\PYG{n+nd}{hover} \PYG{p}{\PYGZob{}}
        \PYG{n+na}{background}\PYG{o}{:} \PYG{l+m+mh}{\PYGZsh{}7EB7C6}\PYG{p}{;}
    \PYG{p}{\PYGZcb{}}
\PYG{p}{\PYGZcb{}}
\end{Verbatim}

If you try to compile this with sass, than the output will be empty. The reason is that you don't use the mixin. And the second thing what you can see at definition of mixin is \emph{\&:hover}. We met with it in nesting, but there we know who is parent. And here it is similar. At the moment when we call the mixin it would have some parent at that will be used.

For inserting the content of the mixin use \emph{@include} directive.

\begin{Verbatim}[commandchars=\\\{\}]
\PYG{n+nc}{.}\PYG{n+nc}{button} \PYG{p}{\PYGZob{}}
    \PYG{k}{@include}\PYG{n+nd}{ button}\PYG{p}{;}
    \PYG{n+na}{height}\PYG{o}{:} \PYG{l+m+mi}{30}\PYG{k+kt}{px}\PYG{p}{;}
\PYG{p}{\PYGZcb{}}
\end{Verbatim}

\begin{Verbatim}[commandchars=\\\{\}]
\PYG{n+nc}{.button} \PYG{p}{\PYGZob{}}
    \PYG{k}{border}\PYG{o}{:} \PYG{k}{thin} \PYG{k}{solid} \PYG{l+m}{\PYGZsh{}40AECA}\PYG{p}{;}
    \PYG{k}{background}\PYG{o}{:} \PYG{l+m}{\PYGZsh{}85C7D8}\PYG{p}{;}
    \PYG{k}{border}\PYG{o}{\PYGZhy{}}\PYG{n}{radius}\PYG{o}{:} \PYG{l+m}{5px}\PYG{p}{;}
    \PYG{k}{color}\PYG{o}{:} \PYG{n+nb}{white}\PYG{p}{;}
    \PYG{k}{height}\PYG{o}{:} \PYG{l+m}{30px}\PYG{p}{;}
\PYG{p}{\PYGZcb{}}
\PYG{n+nc}{.button}\PYG{n+nd}{:hover} \PYG{p}{\PYGZob{}}
        \PYG{k}{background}\PYG{o}{:} \PYG{l+m}{\PYGZsh{}7EB7C6}\PYG{p}{;}
    \PYG{p}{\PYGZcb{}}
\end{Verbatim}

But this is not all what comes with mixins. I give you example when you need to have same styled buttons, but with different background colors. You can define the new color after \emph{@include button}, but there comes some repeating work. You must always define the new behavior for the hover state too. All because of using different color. How can we improve it? We can use the arguments that would be passed to the mixin. The best way how to explain it would be with example.

Say that you want to create three different color buttons. One would be normal with light gray background and would be for classic actions. Next on would be the error button that has red background and finally information button with blue background. And we want to define one mixin and then change just colors when we include it.

\begin{Verbatim}[commandchars=\\\{\}]
\PYG{k}{@mixin}\PYG{n+nf}{ button}\PYG{p}{(}\PYG{n+nv}{\PYGZdl{}color}\PYG{p}{)} \PYG{p}{\PYGZob{}}
    \PYG{n+na}{border}\PYG{o}{:} \PYG{n+no}{thin} \PYG{n+no}{solid} \PYG{n+nv}{\PYGZdl{}color} \PYG{o}{\PYGZhy{}} \PYG{l+m+mh}{\PYGZsh{}222222}\PYG{p}{;}
    \PYG{n+na}{background}\PYG{o}{:} \PYG{n+nv}{\PYGZdl{}color}\PYG{p}{;}
    \PYG{n+na}{border\PYGZhy{}radius}\PYG{o}{:} \PYG{l+m+mi}{5}\PYG{k+kt}{px}\PYG{p}{;}
    \PYG{n+na}{color}\PYG{o}{:} \PYG{n+nb}{white}\PYG{p}{;}
    \PYG{n+na}{padding}\PYG{o}{:} \PYG{l+m+mi}{5}\PYG{k+kt}{px}\PYG{p}{;}
    \PYG{k}{\PYGZam{}}\PYG{n+nd}{:}\PYG{n+nd}{hover} \PYG{p}{\PYGZob{}}
        \PYG{n+na}{background}\PYG{o}{:} \PYG{n+nv}{\PYGZdl{}color} \PYG{o}{\PYGZhy{}} \PYG{l+m+mh}{\PYGZsh{}161616}\PYG{p}{;}
    \PYG{p}{\PYGZcb{}}
\PYG{p}{\PYGZcb{}}

\PYG{n+nc}{.}\PYG{n+nc}{button} \PYG{p}{\PYGZob{}}
    \PYG{k}{@include}\PYG{n+nd}{ button}\PYG{p}{(}\PYG{l+m+mh}{\PYGZsh{}B1B1B1}\PYG{p}{)}\PYG{p}{;}
\PYG{p}{\PYGZcb{}}

\PYG{n+nc}{.}\PYG{n+nc}{error\PYGZhy{}button} \PYG{p}{\PYGZob{}}
    \PYG{k}{@include}\PYG{n+nd}{ button}\PYG{p}{(}\PYG{l+m+mh}{\PYGZsh{}FB4242}\PYG{p}{)}\PYG{p}{;}
\PYG{p}{\PYGZcb{}}

\PYG{n+nc}{.}\PYG{n+nc}{info\PYGZhy{}button} \PYG{p}{\PYGZob{}}
    \PYG{k}{@include}\PYG{n+nd}{ button}\PYG{p}{(}\PYG{l+m+mh}{\PYGZsh{}549EE5}\PYG{p}{)}\PYG{p}{;}
\PYG{p}{\PYGZcb{}}
\end{Verbatim}

\begin{Verbatim}[commandchars=\\\{\}]
\PYG{n+nc}{.button} \PYG{p}{\PYGZob{}}
  \PYG{k}{border}\PYG{o}{:} \PYG{k}{thin} \PYG{k}{solid} \PYG{l+m}{\PYGZsh{}8f8f8f}\PYG{p}{;}
  \PYG{k}{background}\PYG{o}{:} \PYG{l+m}{\PYGZsh{}b1b1b1}\PYG{p}{;}
  \PYG{k}{border}\PYG{o}{\PYGZhy{}}\PYG{n}{radius}\PYG{o}{:} \PYG{l+m}{5px}\PYG{p}{;}
  \PYG{k}{color}\PYG{o}{:} \PYG{n+nb}{white}\PYG{p}{;}
  \PYG{k}{padding}\PYG{o}{:} \PYG{l+m}{5px}\PYG{p}{;} \PYG{p}{\PYGZcb{}}
  \PYG{n+nc}{.button}\PYG{n+nd}{:hover} \PYG{p}{\PYGZob{}}
    \PYG{k}{background}\PYG{o}{:} \PYG{l+m}{\PYGZsh{}9b9b9b}\PYG{p}{;} \PYG{p}{\PYGZcb{}}

\PYG{n+nc}{.error\PYGZhy{}button} \PYG{p}{\PYGZob{}}
  \PYG{k}{border}\PYG{o}{:} \PYG{k}{thin} \PYG{k}{solid} \PYG{l+m}{\PYGZsh{}d92020}\PYG{p}{;}
  \PYG{k}{background}\PYG{o}{:} \PYG{l+m}{\PYGZsh{}fb4242}\PYG{p}{;}
  \PYG{k}{border}\PYG{o}{\PYGZhy{}}\PYG{n}{radius}\PYG{o}{:} \PYG{l+m}{5px}\PYG{p}{;}
  \PYG{k}{color}\PYG{o}{:} \PYG{n+nb}{white}\PYG{p}{;}
  \PYG{k}{padding}\PYG{o}{:} \PYG{l+m}{5px}\PYG{p}{;} \PYG{p}{\PYGZcb{}}
  \PYG{n+nc}{.error\PYGZhy{}button}\PYG{n+nd}{:hover} \PYG{p}{\PYGZob{}}
    \PYG{k}{background}\PYG{o}{:} \PYG{l+m}{\PYGZsh{}e52c2c}\PYG{p}{;} \PYG{p}{\PYGZcb{}}

\PYG{n+nc}{.info\PYGZhy{}button} \PYG{p}{\PYGZob{}}
  \PYG{k}{border}\PYG{o}{:} \PYG{k}{thin} \PYG{k}{solid} \PYG{l+m}{\PYGZsh{}327cc3}\PYG{p}{;}
  \PYG{k}{background}\PYG{o}{:} \PYG{l+m}{\PYGZsh{}549ee5}\PYG{p}{;}
  \PYG{k}{border}\PYG{o}{\PYGZhy{}}\PYG{n}{radius}\PYG{o}{:} \PYG{l+m}{5px}\PYG{p}{;}
  \PYG{k}{color}\PYG{o}{:} \PYG{n+nb}{white}\PYG{p}{;}
  \PYG{k}{padding}\PYG{o}{:} \PYG{l+m}{5px}\PYG{p}{;} \PYG{p}{\PYGZcb{}}
  \PYG{n+nc}{.info\PYGZhy{}button}\PYG{n+nd}{:hover} \PYG{p}{\PYGZob{}}
    \PYG{k}{background}\PYG{o}{:} \PYG{l+m}{\PYGZsh{}3e88cf}\PYG{p}{;} \PYG{p}{\PYGZcb{}}
\end{Verbatim}

Knowledge of this techniques is enough for you to start using the Sass on daily basis. There exists some more advanced things that comes handy, but their main purpose is for make more flexible code that can be part of framework like Compass. If you work on large projects and you use some styling techniques often, than it could be good idea to invest some time to write them into simple framework for you and use it in your projects. For that I recommend for you to continue reading this tutorial. But before you start writing everything on your own, it could be good idea to jump to the chapters about Compass and look if things that you need do exist in it.


\subsection{Control Directives}
\label{src/sass:control-directives}
SassScript supports control directives for including styles only under specific condition or including same style several times with variations. Their main purpose is to use them in mixins, those that are part libraries like Compass and requires flexibility.

\textbf{{}`@if{}`}

\emph{IF} is one of the basics directives for control the flow. The style would be applied only if the condition returns anything else than \emph{false} or \emph{null}.  In conditions are allowed logical operations \emph{and} and \emph{or} that require at least two conditions and the negation \emph{not}.

\begin{Verbatim}[commandchars=\\\{\}]
\PYG{n+nt}{p} \PYG{p}{\PYGZob{}}
    \PYG{k}{@if} \PYG{l+m+mi}{1} \PYG{o}{+} \PYG{l+m+mi}{1} \PYG{o}{==} \PYG{l+m+mi}{2} \PYG{p}{\PYGZob{}} \PYG{n+na}{border}\PYG{o}{:} \PYG{l+m+mi}{1}\PYG{k+kt}{px} \PYG{n+no}{solid}\PYG{p}{;} \PYG{p}{\PYGZcb{}}
    \PYG{k}{@if} \PYG{n+nf}{not}\PYG{p}{(}\PYG{l+m+mi}{5} \PYG{o}{\PYGZgt{}} \PYG{l+m+mi}{3}\PYG{p}{)}          \PYG{p}{\PYGZob{}} \PYG{n+na}{border}\PYG{o}{:} \PYG{l+m+mi}{2}\PYG{k+kt}{px} \PYG{n+no}{dotted}\PYG{p}{;} \PYG{p}{\PYGZcb{}}
    \PYG{k}{@if} \PYG{n}{null}             \PYG{p}{\PYGZob{}} \PYG{n+na}{border}\PYG{o}{:} \PYG{l+m+mi}{4}\PYG{k+kt}{px} \PYG{n+no}{dashed}\PYG{p}{;}\PYG{p}{\PYGZcb{}}
\PYG{p}{\PYGZcb{}}
\end{Verbatim}

\begin{Verbatim}[commandchars=\\\{\}]
\PYG{n+nt}{p} \PYG{p}{\PYGZob{}} \PYG{k}{border}\PYG{o}{:} \PYG{l+m}{1px} \PYG{k}{solid}\PYG{p}{;} \PYG{p}{\PYGZcb{}}
\end{Verbatim}

At case that you need to check if the variable content is one of many, than comes handy the \emph{@else if}. The last must be \emph{@else}.

\begin{Verbatim}[commandchars=\\\{\}]
\$language: ruby;
p \PYGZob{}
    @if \$language == python \PYGZob{}
        background: green;
    \PYGZcb{} @else if \$language == c\# \PYGZob{}
        background: blue;
    \PYGZcb{} @else if \$language == ruby \PYGZob{}
        background: red;
    \PYGZcb{} @else \PYGZob{}
        background: yellow;
    \PYGZcb{}
\PYGZcb{}
\end{Verbatim}

\begin{Verbatim}[commandchars=\\\{\}]
\PYG{n+nt}{p} \PYG{p}{\PYGZob{}} \PYG{k}{background}\PYG{o}{:} \PYG{n+nb}{red}\PYG{p}{;} \PYG{p}{\PYGZcb{}}
\end{Verbatim}

\textbf{{}`@for{}`}

In case that you need to repeat some action with different value in the output, you can use \emph{@for} cycle. It sets the value in variable from starting point to end. There are two forms of for-cycle in Sass. First is \emph{@for \$var from \textless{}start\textgreater{} through \textless{}end\textgreater{}} and the second is \emph{@for \$var from \textless{}start\textgreater{} to \textless{}end\textgreater{}}. The variable \emph{\$var} is normal variable that can be named how you need. It's common to name it \emph{\$i}. The \emph{\textless{}start\textgreater{}} and \emph{\textless{}end\textgreater{}} can be any expressions that returns integer. The difference between these two forms is in the \emph{trough} and \emph{to}. If you use \emph{through} the \emph{\textless{}end\textgreater{}} value will be used at the end. If you use \emph{to} the cycle stops at the \emph{\textless{}end\textgreater{}-1} value.

\begin{Verbatim}[commandchars=\\\{\}]
\PYG{k}{@for} \PYG{n+nv}{\PYGZdl{}i} \PYG{o+ow}{from} \PYG{l+m+mi}{1} \PYG{o+ow}{to} \PYG{l+m+mi}{4} \PYG{p}{\PYGZob{}}
    \PYG{n+nc}{.}\PYG{n+nc}{item\PYGZhy{}}\PYG{l+s+si}{\PYGZsh{}\PYGZob{}}\PYG{n+nv}{\PYGZdl{}i}\PYG{l+s+si}{\PYGZcb{}} \PYG{p}{\PYGZob{}} \PYG{n+na}{width}\PYG{o}{:}  \PYG{l+m+mi}{2}\PYG{k+kt}{em} \PYG{o}{*} \PYG{n+nv}{\PYGZdl{}i}\PYG{p}{;} \PYG{p}{\PYGZcb{}}
\PYG{p}{\PYGZcb{}}
\end{Verbatim}

\begin{Verbatim}[commandchars=\\\{\}]
\PYG{n+nc}{.item\PYGZhy{}1} \PYG{p}{\PYGZob{}} \PYG{k}{width}\PYG{o}{:} \PYG{l+m}{2em}\PYG{p}{;} \PYG{p}{\PYGZcb{}}
\PYG{n+nc}{.item\PYGZhy{}2} \PYG{p}{\PYGZob{}} \PYG{k}{width}\PYG{o}{:} \PYG{l+m}{4em}\PYG{p}{;} \PYG{p}{\PYGZcb{}}
\PYG{n+nc}{.item\PYGZhy{}3} \PYG{p}{\PYGZob{}} \PYG{k}{width}\PYG{o}{:} \PYG{l+m}{6em}\PYG{p}{;} \PYG{p}{\PYGZcb{}}
\end{Verbatim}

\textbf{{}`@each{}`}

The for cycle is good if you are working with numbers. But if you want to work with list of words, than using the \emph{@each} is better decision.  The syntax for each is simple. \emph{@each \$var in \textless{}list\textgreater{}}. The variable \emph{\$var} is working the same way how in the for-cycle. So in every step the value


\chapter{Compass for adventurer}
\label{src/compass_begin::doc}\label{src/compass_begin:compass-for-adventurer}
At this point you should know enough about Sass and what is possible to do in it. There are many ways how to use it. You can use the approach of writing everything for yourself and only if you need it. There are probably some of you who always work this way. But many developers want to save so much time that it's possible so they can start with another project. They often take some framework, that has the common things written and they just write the new parts specific for their project. The community about Sass is not different. There is not just one framework that you can use, but in this tutorial I will talk about the oldest and probably the most used of them. Compass. It's created by one of the authors of the Sass.


\section{Installation}
\label{src/compass_begin:installation}
Installation of this framework, so you can use it in your project, is simple enough. The important thing is to have installed Sass. If you don't, for some reason, than go to first chapter where is written the guide for installation. Just put into console following command.:

gem install compass

It will download and set up path to compass files.

Compass creates prepared mixins and functions that extends the functionality of the Sass.


\section{Options}
\label{src/compass_begin:options}

\chapter{CSS3 and many browsers? With Compass easy}
\label{src/compass_css3:css3-and-many-browsers-with-compass-easy}\label{src/compass_css3::doc}
The CSS3 brings many new features to modern browser. From basic things like border radius through box-shadow to advance effects created with transitions and keyframes. Everything would be great if CSS3 would be completed and all features in it implemented in all browsers the same way. But this is just wish of all web-developers. Truth is that browsers came with their own prefixes for function of CSS3 which has not yet been accepted to the final version or they have their own ideas that they want to make their browser more advance for example.

Every web-developer today must have seen vendor prefixes like -webkit-, -ms-, -mz- and -o-. But what if you need to create button that drops shadow. In perfect universe you will write

\begin{Verbatim}[commandchars=\\\{\}]
\PYG{n+nc}{.perfect\PYGZhy{}button} \PYG{p}{\PYGZob{}}
    \PYG{o}{...}
    \PYG{n}{box}\PYG{o}{\PYGZhy{}}\PYG{n}{shadow}\PYG{o}{:} \PYG{l+m}{2px} \PYG{l+m}{2px} \PYG{l+m}{5px} \PYG{l+m}{4px} \PYG{n}{rgba}\PYG{p}{(}\PYG{l+m}{42}\PYG{o}{,}\PYG{l+m}{42}\PYG{o}{,}\PYG{l+m}{42}\PYG{o}{,}\PYG{l+m}{0}\PYG{o}{.}\PYG{l+m}{8}\PYG{p}{);}
    \PYG{o}{...}
\PYG{p}{\PYGZcb{}}
\end{Verbatim}

But in this world you must write something like

\begin{Verbatim}[commandchars=\\\{\}]
\PYG{n+nc}{.just\PYGZhy{}button} \PYG{p}{\PYGZob{}}
    \PYG{o}{...}
    \PYG{o}{\PYGZhy{}}\PYG{n}{webkit}\PYG{o}{\PYGZhy{}}\PYG{n}{box}\PYG{o}{\PYGZhy{}}\PYG{n}{shadow}\PYG{o}{:} \PYG{l+m}{2px} \PYG{l+m}{2px} \PYG{l+m}{5px} \PYG{l+m}{4px} \PYG{n}{rgba}\PYG{p}{(}\PYG{l+m}{42}\PYG{o}{,}\PYG{l+m}{42}\PYG{o}{,}\PYG{l+m}{42}\PYG{o}{,}\PYG{l+m}{0}\PYG{o}{.}\PYG{l+m}{8}\PYG{p}{);}
       \PYG{o}{\PYGZhy{}}\PYG{n}{moz}\PYG{o}{\PYGZhy{}}\PYG{n}{box}\PYG{o}{\PYGZhy{}}\PYG{n}{shadow}\PYG{o}{:} \PYG{l+m}{2px} \PYG{l+m}{2px} \PYG{l+m}{5px} \PYG{l+m}{4px} \PYG{n}{rgba}\PYG{p}{(}\PYG{l+m}{42}\PYG{o}{,}\PYG{l+m}{42}\PYG{o}{,}\PYG{l+m}{42}\PYG{o}{,}\PYG{l+m}{0}\PYG{o}{.}\PYG{l+m}{8}\PYG{p}{);}
            \PYG{n}{box}\PYG{o}{\PYGZhy{}}\PYG{n}{shadow}\PYG{o}{:} \PYG{l+m}{2px} \PYG{l+m}{2px} \PYG{l+m}{5px} \PYG{l+m}{4px} \PYG{n}{rgba}\PYG{p}{(}\PYG{l+m}{42}\PYG{o}{,}\PYG{l+m}{42}\PYG{o}{,}\PYG{l+m}{42}\PYG{o}{,}\PYG{l+m}{0}\PYG{o}{.}\PYG{l+m}{8}\PYG{p}{);}
    \PYG{o}{...}
\PYG{p}{\PYGZcb{}}
\end{Verbatim}

This is one of the better cases when there not all vendors have their own prefixes. But we need to add two more lines just to secure that it will work in as many browsers as possible. And finally the same example written in Sass using Compass.

\begin{Verbatim}[commandchars=\\\{\}]
\PYG{k}{@import} \PYG{l+s+s2}{\PYGZdq{}}\PYG{l+s+s2}{compass/css3}\PYG{l+s+s2}{\PYGZdq{}}

\PYG{o}{.}\PYG{n}{scss\PYGZhy{}button} \PYG{p}{\PYGZob{}}
    \PYG{n+nc}{.}\PYG{n+nc}{.}\PYG{n+nc}{.}
    \PYG{o}{@}\PYG{n+nt}{include} \PYG{n+nt}{box\PYGZhy{}shadow}\PYG{o}{(}\PYG{n+nt}{rgba}\PYG{o}{(}\PYG{n+nt}{42}\PYG{o}{,}\PYG{n+nt}{42}\PYG{o}{,}\PYG{n+nt}{42}\PYG{o}{,}\PYG{n+nt}{0}\PYG{n+nc}{.}\PYG{n+nc}{8}\PYG{o}{)} \PYG{n+nt}{2px} \PYG{n+nt}{2px} \PYG{n+nt}{5px} \PYG{n+nt}{4px}\PYG{o}{)}\PYG{o}{;}
    \PYG{n+nc}{.}\PYG{n+nc}{.}\PYG{n+nc}{.}
\PYG{p}{\PYGZcb{}}
\end{Verbatim}

The import is needed only once so I will not count it. We are again at one line for the box-shadow. The code with vendor prefixes will be generated by mixin defined in Compass.


\chapter{More Compass features}
\label{src/compass_utilities:more-compass-features}\label{src/compass_utilities::doc}
The title and content of this chaper will posibly change in future.


\chapter{World with Internet Explorers}
\label{src/compass_pie:world-with-internet-explorers}\label{src/compass_pie::doc}
Do you like Internet Explorer (8 and olders)? If yes than I don't think that you are web-designer. The number of active old versions of Internet Explorer today is till large. And when you are working on new breath taking design of your site you want to use all features to make it more attractive for visitors. But than comes at place user with old version of Internet Explorer. Etc. IE6. And your perfectly designed site using CSS3 is not working properly.

Here should come more about it. But I don't have prepared anything for now. So it would be blank. For now.


\chapter{Tips for writing large projects using SCSS}
\label{src/cookbook_tips::doc}\label{src/cookbook_tips:tips-for-writing-large-projects-using-scss}
\begin{notice}{note}{Todo}

\href{https://speakerd.s3.amazonaws.com/presentations/620348f03c340130dcdc12313918315a/Sass\_junk-drawer.pdf}{Sass\_junk-drawer.pdf} (https://speakerd.s3.amazonaws.com/presentations/620348f03c340130dcdc12313918315a/Sass\_junk-drawer.pdf) source for some info
\end{notice}

In this chapter I would like to mention my own ideas and steps to write nice SCSS code of large projects, that probably had more than just one guy for CSS.

I don't think there are some standarts how to write code in Sass (respectively in CSS). At least I think that every larger project should have created some points how the SCSS files would be written. It can save a lot of time in cases when you need to change some parts of your design that you haven't seen for weeks or they were written by somebody else.


\section{Writing style}
\label{src/cookbook_tips:writing-style}

\section{Naming conventions}
\label{src/cookbook_tips:naming-conventions}

\section{Splitting into files}
\label{src/cookbook_tips:splitting-into-files}

\chapter{Layouts code examples using Sass features}
\label{src/cookbook_layout:layouts-code-examples-using-sass-features}\label{src/cookbook_layout::doc}

\chapter{Menu snippets}
\label{src/cookbook_menu:menu-snippets}\label{src/cookbook_menu::doc}

\chapter{Form elements}
\label{src/cookbook_form::doc}\label{src/cookbook_form:form-elements}

\section{Inputs}
\label{src/cookbook_form:inputs}

\section{Buttons}
\label{src/cookbook_form:buttons}

\section{Dropdowns}
\label{src/cookbook_form:dropdowns}

\chapter{Conclusion}
\label{src/conclusion::doc}\label{src/conclusion:conclusion}
So here would be some words about my work.


\chapter{Indices and tables}
\label{index:indices-and-tables}\begin{itemize}
\item {} 
\emph{genindex}

\item {} 
\emph{modindex}

\item {} 
\emph{search}

\end{itemize}

\begin{notice}{note}{Todo}

\href{https://speakerd.s3.amazonaws.com/presentations/620348f03c340130dcdc12313918315a/Sass\_junk-drawer.pdf}{Sass\_junk-drawer.pdf} (https://speakerd.s3.amazonaws.com/presentations/620348f03c340130dcdc12313918315a/Sass\_junk-drawer.pdf) source for some info
\end{notice}

(The {\hyperref[src/cookbook_tips:index-0]{\emph{original entry}}} is located in  /home/darjanin/Dropbox/thesis/work/src/cookbook\_tips.rst, line 4.)

\begin{notice}{note}{Todo}

This text possibly will be rewritten cause I figure out that on Mac is old version of Ruby.
\end{notice}

(The {\hyperref[src/sass:index-0]{\emph{original entry}}} is located in  /home/darjanin/Dropbox/thesis/work/src/sass.rst, line 48.)

\begin{notice}{note}{Todo}

rethink this section if it's important. many of these things are used if you write your own mixins.
\end{notice}

(The {\hyperref[src/sass:index-1]{\emph{original entry}}} is located in  /home/darjanin/Dropbox/thesis/work/src/sass.rst, line 287.)

\begin{notice}{note}{Todo}

at this point I need to make some example files and test how it would behave.
\end{notice}

(The {\hyperref[src/sass:index-2]{\emph{original entry}}} is located in  /home/darjanin/Dropbox/thesis/work/src/sass.rst, line 526.)

\begin{notice}{note}{Todo}

This text possibly will be rewritten cause I figure out that on Mac is old version of Ruby.
\end{notice}

(The \emph{original entry} is located in  /home/darjanin/Dropbox/thesis/work/src/sass\_begin.rst, line 45.)



\renewcommand{\indexname}{Index}
\printindex
\end{document}
